\documentclass[10pt, twocolumn]{article}
\usepackage{amsmath}
\usepackage{setspace}
\title{CS2100 Summary}

\setlength{\columnsep}{20pt}

\begin{document}
\begin{center}
\part*{CS2100 Summary}
\end{center}
\tableofcontents
\newpage
\onehalfspacing

\section{Number Systems and Code}
\subsection{Number Systems}
\subsubsection{Base Conversion}
{\bf Base-R to Decimal Conversion}
\newline
Sum up digits multiplied by their weights:
\newline
$1101.101_2 = 1\times 2^3 + 1\times 2^2 + 1\times 2^0 + 1\times 2^{-1} + 1\times 2^{-3}$ \\
$572.6_8 = 5\times 8^2 + 7\times 8^1 + 2\times 8^0 + 6\times 8^{-1}$
\newline
{\bf Decimal to Binary Conversion}
\begin{enumerate}
\item[1.]{Sum-of-Weights Method} \\
Determine the set of binary weights that sum up to the decimal number.
\newline
$9_{10} = 8+1 = 2^3+2^0 = 1001_2$ \\
$18_{10} = 16+2 = 2^4+2^1 = 10100_2$
\item[2.]{Repeated Division/Multiplication} \\
For whole numbers (Division by 2) until denominator is $0$:
\begin{center}
$\begin{array}{c|lll}
2 & 43 &\\
2 & 21 & \text{rem} & 1 \leftarrow \text{LSB} \\
2 & 10 & \text{rem} & 1 \\
2 & 5 & \text{rem} & 0 \\
2 & 2 & \text{rem} & 1 \\
2 & 1 & \text{rem} & 0 \\
  & 0 & \text{rem} & 1 \leftarrow \text{MSB}
\end{array}$
\end{center}
Form binary number starting from MSB (Most Siginificant Bit):
\begin{center}
$43_{10} = 101011_2$
\end{center}
For fractional numbers (Multiplication by 2) until result is $1$:
\begin{center}
$\begin{array}{l|l}
 & \text{Carry} \\ \hline
0.3125\times 2=0.625 & 0 \leftarrow \text{MSB} \\
0.625\times 2=1.25 & 1 \\
0.25\times 2=0.50 & 0 \\
0.5\times 2=1.00 & 1 \leftarrow \text{LSB}
\end{array}$
\end{center}
Form binary number starting from MSB (Most Significant Bit):
\begin{center}
$0.3125 = 0.0101_2$
\end{center}
\end{enumerate}
{\bf Binary to Octal/Hexadecimal Conversion}
\begin{enumerate}
\item[1.]{Binary $\rightarrow$ Octal} \\
Partition in groups of 3 \\ 
Whole: R to L \hspace{20pt} Fractional: L to R
\begin{center}
$(10\text{ }111\text{ }011\text{ }001\text{ . }101\text{ }110)_2 = (2751.56)_8$
\end{center}
\item[2.]{Octal $\rightarrow$ Binary} \\
Convert each digit into 3-bit binary and concatenate.
\item[3.]{Binary $\rightarrow$ Hexadecimal} \\
Partition in groups of 4 \\
Whole: R to L \hspace{20pt} Fractional: L to R
\begin{center}
$(101\text{ }1101\text{ }1001\text{ . }1011\text{ }1000)_2 = (5D9.B8)_{16}$
\end{center}
\item[4.]{Hexadecimal $\rightarrow$ Binary} \\
Convert each digit into 4-bit binary and concatenate.
\end{enumerate}

\subsubsection{Negative Numbers}
3 Common Representations:
\begin{enumerate}
\item[1.]{Sign-and-Magnitude}
\item[2.]{1s Complement}
\item[3.]{2s Complement}
\end{enumerate}
{\bf Sign-and-Magnitude $(xxx)_{sm}$}
\begin{itemize}
\item 1-bit sign and 7-bit magnitude
\item Negation: \\
Invert left most bit
\end{itemize}
{\bf 1s Complement $(xxx)_{1s}$}
\newline
Given $x$ as a $n$-bit number, its negated form in 1s complement:
\begin{center}
$-x = 2^n-x-1$ (calculation in decimal)
\end{center}
\begin{itemize}
\item MSB still represents sign
\item Negation: \\
Invert all bits of positive binary form
\end{itemize}
{\bf 2s Complement $(xxx)_{2s}$}
\newline
Given $x$ as a $n$-bit number, its negated form in 2s complement:
\begin{center}
$-x = 2^n-x$ (calculation in decimal)
\end{center}
\begin{itemize}
\item MSB still represents sign
\item Negation: \\
Invert all bits of positive binary form then add 1 to LSB
\end{itemize}
{\bf 2s Complement Addition/Subtraction}
\begin{enumerate}
\item[1.]{Addition $A+B$}
\begin{itemize}
\item Perform binary addition
\item Ignore carry out of MSB
\item Check for overflow (if result is opposite sign of $A$ and $B$)
\end{itemize}
\item[2.]{Subtraction $A-B$}
\newline
$A-B = A+(-B)$
\begin{itemize}
\item Take 2s-complement of B
\item Add the 2s-complement of B to A
\end{itemize}
\end{enumerate}
{\bf 1s Complement Addition/Subtraction}
\begin{enumerate}
\item[1.]{Addition $A+B$}
\begin{itemize}
\item Perform binary addition
\item If there is carry out on MSB, add one to end result
\item Check for overflow (if result is opposite sign of $A$ and $B$)
\end{itemize}
\item[2.]{Subtraction $A-B$}
\newline
$A-B = A+(-B)$
\begin{itemize}
\item Take 1s-complement of $B$
\item Add the 1s-complement of $B$ to $A$.
\end{itemize}
\end{enumerate}
{\bf Excess Representation}
\begin{itemize}
\item Given Excess-$n$ representation
\item Add $n$ to decimal value
\item Convert result to binary
\end{itemize}

\subsubsection{Floating Point Numbers}
Consists of: \\
\begin{center}
$\begin{array}{|c|c|c|}
\text{Sign} & \text{Mantissa} & \text{Exponent}
\end{array}$
\end{center}
{\bf Sign}
\begin{itemize}
\item 0 for positive
\item 1 for negative
\end{itemize}
{\bf Mantissa}
\begin{itemize}
\item Normalised form
\item $0.01101 \times 2^4 \rightarrow \text{normalised} \rightarrow 0.1101 \times 2^3$
\item $101011.0110 \times 2^{-4} \rightarrow \text{normalised} \rightarrow 0.1010110110 \times 2^2








\end{document}